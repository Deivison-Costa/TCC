A realização deste trabalho apresentou um sistema  para monitoramento agrícola em tempo real, abordando as necessidades específicas de manejo hídrico e climático para culturas agrícolas. O desenvolvimento da solução baseou-se em boas práticas de engenharia de \textit{software} e \textit{hardware}, garantindo uma integração consistente entre os componentes físicos e computacionais. Além disso, o sistema permitiu a coleta de dados ambientais relevantes, como temperatura, umidade, radiação solar, pressão atmosférica e velocidade do vento, utilizados para calcular a evapotranspiração (ETo e ETc) com base no método FAO-56 \parencite{Allen_evapotranspiration1998}.

Os resultados obtidos demonstraram que o sistema atende aos objetivos propostos, especialmente no que tange à usabilidade, flexibilidade e capacidade de adaptação a diferentes cenários agrícolas. A interface intuitiva contribuiu para a acessibilidade e compreensão dos dados pelos usuários finais, enquanto a modularidade da solução oferece espaço para melhorias e expansões futuras.

Contudo, algumas limitações foram identificadas, como a necessidade de calibração mais frequente do sensor de umidade do solo e a dependência de conectividade \textit{Wi-Fi} no raio de acesso. Apesar dessas questões, o sistema mostrou-se promissor e funcional dentro das condições de teste na microcultura observada, com potencial para aplicação em escalas maiores.

Em suma, este trabalho apresenta uma contribuição para o manejo sustentável da agricultura, ao propor uma ferramenta prática e acessível que alia tecnologia e ciência no auxílio à tomada de decisões em campo. As perspectivas de evolução do sistema, descritas na próxima seção, reforçam a viabilidade de ampliação do impacto deste projeto para diferentes cenários e culturas agrícolas.