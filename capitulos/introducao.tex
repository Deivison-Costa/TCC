Na presente seção, apresentam-se uma contextualização do tema e, posteriormente, a justificativa e os objetivos. Por fim, a estrutura do trabalho é mostrada.

\section{Contextualização do tema}
A agricultura é uma atividade fundamental para a economia global, sendo responsável pela produção de alimentos e matérias-primas essenciais para a sobrevivência e o bem-estar da humanidade. De acordo \textcite{carmody_fao2023}, o setor agrícola enfrenta desafios significativos. A crescente demanda por alimentos, devido ao aumento populacional e às mudanças climáticas, afeta a disponibilidade de recursos naturais. Diante desses desafios, a gestão eficiente dos recursos hídricos e a adoção de tecnologias avançadas tornam-se cruciais para garantir a sustentabilidade e a produtividade das culturas \parencite{carmody_fao2023}.

No Brasil, a agricultura desempenha um papel vital na economia, sendo uma das principais fontes de crescimento e desenvolvimento do país \parencite{Ramos_irrigacao2022}. O Brasil é um dos maiores produtores e exportadores de alimentos do mundo, destacando-se na produção de soja, milho, café e carne bovina \parencite{Ramos_irrigacao2022}. 

A irrigação é uma prática que tem sido objeto de estudo no país, visando aumentar a produtividade agrícola e contribuir para a segurança alimentar, conforme aponta \textcite{Ramos_irrigacao2022}. Métodos tradicionais de irrigação, muitas vezes, carecem de precisão e eficiência, levando ao desperdício de água e outros recursos \parencite{Pereira_irrigation2002}.

A evapotranspiração (ET) é uma das principais variáveis a serem consideradas na gestão eficiente da água \parencite{yang_nature2023}, sendo fundamental para determinar as necessidades hídricas das culturas e otimizar o uso da água na agricultura \parencite{carmody_fao2023, yang_nature2023}. Esta variável (ET) representa a perda de água do solo por evaporação e pela transpiração das plantas. 

O desenvolvimento de sistemas de monitoramento em tempo real para a irrigação surge como uma abordagem promissora para otimizar o uso da água e melhorar o rendimento das culturas \parencite{Vijh_system2024, Dutta_system2024, Amine_system2024}. De acordo com \textcite{Prathibha_system2017}, tais sistemas fornecem aos agricultores informações precisas e atualizadas sobre as condições do solo e das plantas, permitindo uma gestão mais eficiente da irrigação ao longo do tempo.

\section{Justificativa}

Este trabalho propõe o desenvolvimento de um sistema \textit{web} para monitoramento de variáveis ambientais, com o objetivo de otimizar a gestão da irrigação na agricultura. Integrando tecnologias de \textit{hardware} e \textit{software}, o sistema coleta e analisa dados climáticos em tempo real, oferecendo informações para a tomada de decisões mais sustentáveis e eficientes.

A adoção de ferramentas tecnológicas para o monitoramento ambiental incentiva a agricultura inteligente \parencite{Garg_smart2023}. Essas tecnologias permitem o ajuste dinâmico das atividades agrícolas com base nas variáveis climáticas coletadas. Isso possibilita aos produtores maximizar a produtividade enquanto reduz os impactos ambientais, especialmente no uso de água \parencite{Gurjeet_smart2022}.

Considerando o alto consumo hídrico da agricultura, são necessárias soluções para otimizar o uso desses recursos \parencite{Ramos_irrigacao2022}. Ferramentas de monitoramento mais precisas podem favorecer uma gestão hídrica eficiente, reduzindo o desperdício e os custos operacionais \parencite{carmody_fao2023}.

Assim, a relevância deste estudo está na contribuição para uma agricultura mais prática e adaptável, que responda às diferentes condições climáticas e de solo, promovendo um modelo de produção sustentável.

\section{Objetivos}

Este trabalho teve como objetivo geral desenvolver um sistema de monitoramento em tempo real para o cultivo agrícola. A proposta foi utilizar tecnologias modernas, como Sistemas Cyberfísicos (CPS), aplicações web e protocolos de comunicação (HTTP e MQTT), com o método da Evapotranspiração da Cultura (ETc), visando aumentar a eficácia do monitoramento da cultura e de suas necessidades hídricas.

Os objetivos específicos deste trabalho foram os seguintes:

\begin{itemize}
    \item Implementar um sistema de telemetria de sensores para coletar dados ambientais-chave em tempo real;
    \item Integrar algoritmos de cálculo da ETc ao sistema para fornecer métricas;
    \item Desenvolver uma interface de usuário intuitiva para facilitar o acesso e a compreensão dos dados pelos agricultores.
\end{itemize}

\section{Estrutura do trabalho}
Este trabalho está estruturado da seguinte forma:

\begin{itemize}
    \item O Capítulo 2 apresenta os conceitos e teorias que servem como base para o desenvolvimento deste trabalho, além de fornecer o estado da arte da área de estudo;
    \item O Capítulo 3 descreve os procedimentos, técnicas e ferramentas utilizadas para a condução do trabalho;
    \item O Capítulo 4 expõe os principais achados do estudo, analisando os dados obtidos e destacando suas implicações;
    \item O Capítulo 5 apresenta a conclusão do trabalho, discutindo os resultados obtidos e as contribuições do estudo.
    \item Por fim, o Capítulo 6 traz possíveis direções futuras para a continuidade e o aprimoramento do sistema proposto.
\end{itemize}
